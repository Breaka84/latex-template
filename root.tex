% This is LLNCS.DEM the demonstration file of
% the LaTeX macro package from Springer-Verlag
% for Lecture Notes in Computer Science,
% version 2.4 for LaTeX2e as of 16. April 2010
%
\documentclass[12pt]{llncs} %%%%%%%%%%%%%%%%%%%%% Used for drafting %%%%%%%%%%%%%%%%%%%%%
%\documentclass{llncs}
%

\usepackage[colorlinks, linkcolor = black, citecolor = black, filecolor = black, urlcolor = blue]{hyperref} 

% APA Cite Package and initialization
%\RequirePackage{apacite}
\usepackage[apaciteclassic]{apacite}
\bibliographystyle{apacite}

%
\usepackage{makeidx}  % allows for indexgeneration
%
\usepackage{color}
% 
\setcounter{tocdepth}{2}
% 
% make a proper TOC despite llncs
\makeatletter
\renewcommand*\l@author[2]{}
\renewcommand*\l@title[2]{}
\makeatletter

\usepackage[UTF8]{inputenc}
%\usepackage[T1]{fontenc}
%\usepackage[ngerman]{babel}

\usepackage{comment}

\usepackage{tabularx}



\linespread{1.6} %%%%%%%%%%%%%%%%%%%%% Used for drafting %%%%%%%%%%%%%%%%%%%%%
\usepackage[margin=2.5cm]{geometry} %%%%%%%%%%%%%%%%%%%%% Used for drafting %%%%%%%%%%%%%%%%%%%%%

%=====================================================================================

\begin{document} 
%
\frontmatter          % for the preliminaries 
%
\pagestyle{headings}  % switches on printing of running heads
% \addtocmark{Hamiltonian Mechanics} % additional mark in the TOC
%

\author{David Eigenstuhler BA}
\title{Augmented Reality Support for Business Process Modeling}
\titlerunning{Augmented Reality in BPM}  % abbreviated title (for running head)
%                                     also used for the TOC unless
%                                     \toctitle is used
%\subtitle{Proposal}
\institute{Communications Engineering\\
Department of Business Information Systems\\
Johannes Kepler University of Linz}
\maketitle

%\vspace{5cm} 
\begin{flushright}\noindent
April 2014 
\end{flushright}
\vspace{1cm}

%\begin{tabular}{p{8cm}p{8cm}}
%\begin{tabular}{>{\centering\arraybackslash}p{4cm}p{4cm}p{4cm}}
\begin{tabular}{p{4cm}p{4cm}p{4cm}}
Course Supervisor & & Course Supervisor \tabularnewline 
Udo Kannengiesser & & Eric Brewster \tabularnewline 
\end{tabular} 

%=====================================================================================

%\newpage

\vspace{1cm}
\begin{abstract}
Using virtual worlds or Augmented Reality for Business Process Modeling (BPM) is a fairly new approach 
in the process modeling field. Having domain experts with little modeling skills and process modeling
experts with no knowledge of the process to be modeled requires new solutions. This paper examines three
different methods to give an overview of current approaches, which try to overcome this problem by
tangible physical objects, virtual reality, and Augmented Reality respectively. The analysis is done
on the basis of secondary research only. Physical and tactile modeling makes it easy for domain experts 
to interact with the model. The major drawback of the necessary manual transcription of the results is 
being avoided by the idea to include the modeler into virtual worlds. Using the Second Life engine provides
easier immersion into the model than general BPM methods. This solution lacks the natural communication
component such as nonverbal signals, which is overcome by combining the two aforementioned solutions. 
An augmented tabletop featuring physical objects and direct software integration requires in turn special
hardware. There is no perfect method at this time but current methods already provide significant advantages
over traditional BPM procedures.\\
\vspace{1cm}
 
\textbf{Keywords: }
\textit{Business Process Modeling, Augmented Reality, Virtual Reality,
Tangible-BPM, Subject-oriented BPM}
\end{abstract}

%=====================================================================================
%
%\tableofcontents
%
\mainmatter              % start of the contributions
%
\title{Augmented Reality Support for Business Process Modeling}
%
%=====================================================================================

\section{Introduction}

Please have a look at \cite{Kopka2003}!

Lorem ipsum dolor sit amet, consetetur sadipscing elitr, sed diam nonumy eirmod
tempor invidunt ut labore et dolore magna aliquyam erat, sed diam voluptua. At

\subsection{sub1}
vero eos et accusam et justo duo dolores et ea rebum. Stet clita kasd gubergren,
no sea takimata sanctus est Lorem ipsum dolor sit amet. Lorem ipsum dolor sit
amet, consetetur sadipscing elitr, sed diam nonumy eirmod tempor invidunt ut

\subsection{sub2}
labore et dolore magna aliquyam erat, sed diam voluptua. At vero eos et accusam
et justo duo dolores et ea rebum. Stet clita kasd gubergren, no sea takimata
sanctus est Lorem ipsum dolor sit amet.

\section{Context}

Lorem ipsum dolor sit amet, consetetur sadipscing elitr, sed diam nonumy eirmod
tempor invidunt ut labore et dolore magna aliquyam erat, sed diam voluptua. At

\subsection{sub1}
vero eos et accusam et justo duo dolores et ea rebum. Stet clita kasd gubergren,

\subsection{sub2}
no sea takimata sanctus est Lorem ipsum dolor sit amet. Lorem ipsum dolor sit
amet, consetetur sadipscing elitr, sed diam nonumy eirmod tempor invidunt ut
labore et dolore magna aliquyam erat, sed diam voluptua. At vero eos et accusam

\subsection{sub3}
et justo duo dolores et ea rebum. Stet clita kasd gubergren, no
sea takimata sanctus est Lorem ipsum dolor sit amet.

\section{Evidence and Analytical Methods} 

Lorem ipsum dolor sit amet, consetetur sadipscing elitr, sed diam nonumy eirmod
tempor invidunt ut labore et dolore magna aliquyam erat, sed diam voluptua. At
vero eos et accusam et justo duo dolores et ea rebum. Stet clita kasd gubergren,
no sea takimata sanctus est Lorem ipsum dolor sit amet. Lorem ipsum dolor sit
\subsection{sub1}
amet, consetetur sadipscing elitr, sed diam nonumy eirmod
tempor invidunt ut labore et dolore magna aliquyam erat, sed diam voluptua. At vero eos et accusam
et justo duo dolores et ea rebum. Stet clita kasd gubergren, no sea takimata
sanctus est Lorem ipsum dolor sit amet.

\section{Main}

Lorem ipsum dolor sit amet, consetetur sadipscing elitr, sed diam nonumy eirmod
tempor invidunt ut labore et dolore magna aliquyam erat, sed diam voluptua. At
vero eos et accusam et justo duo dolores et ea rebum. Stet clita kasd gubergren,
no sea takimata sanctus est Lorem ipsum dolor sit amet. Lorem ipsum dolor sit
amet, consetetur sadipscing elitr, sed diam nonumy eirmod tempor invidunt ut
labore et dolore magna aliquyam erat, sed diam voluptua. At vero eos et accusam
et justo duo dolores et ea rebum. Stet clita kasd gubergren, no sea takimata
sanctus est Lorem ipsum dolor sit amet.

\section{Conclusion}

Lorem ipsum dolor sit amet, consetetur sadipscing elitr, sed diam nonumy eirmod
tempor invidunt ut labore et dolore magna aliquyam erat, sed diam voluptua. At
vero eos et accusam et justo duo dolores et ea rebum. Stet clita kasd gubergren,
no sea takimata sanctus est Lorem ipsum dolor sit amet. Lorem ipsum dolor sit
amet, consetetur sadipscing elitr, sed diam nonumy eirmod tempor invidunt ut
labore et dolore magna aliquyam erat, sed diam voluptua. At vero eos et accusam
et justo duo dolores et ea rebum. Stet clita kasd gubergren, no sea takimata
sanctus est Lorem ipsum dolor sit amet.



\clearpage




\printindex

\bibliography{refs}

\clearpage

\pagestyle{empty}
\parindent0mm

David Eigenstuhler BA\\
Johannes Kepler University of Linz, Austria\\
\href{mailto:david@eigenstuhler.at}{David@Eigenstuhler.at}\\
\vspace{4mm}

\textbf{Biodata: } David Eigenstuhler is momentarily attending the Master's program in Business Informatics in Linz. He earned his Bachelor of Arts' degree 
in Business Studies. Next to his studies, he works as an IT administrator at Emesa Austria. Before he commenced to study in 2009, 
he worked in product management and sales promotion.

\end{document}
