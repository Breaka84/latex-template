% This is LLNCS.DEM the demonstration file of
% the LaTeX macro package from Springer-Verlag
% for Lecture Notes in Computer Science,
% version 2.4 for LaTeX2e as of 16. April 2010
%
\documentclass[12pt]{llncs} %%%%%%%%%%%%%%%%%%%%% Used for drafting %%%%%%%%%%%%%%%%%%%%%
%\documentclass{llncs}
%

\usepackage[colorlinks, linkcolor = black, citecolor = black, filecolor = black, urlcolor = blue]{hyperref} 

% APA Cite Package and initialization
%\RequirePackage{apacite}
\usepackage[apaciteclassic]{apacite}
\bibliographystyle{apacite}

%
\usepackage{makeidx}  % allows for indexgeneration
%
\usepackage{color}
% 
\setcounter{tocdepth}{2}
% 
% make a proper TOC despite llncs
\makeatletter
\renewcommand*\l@author[2]{}
\renewcommand*\l@title[2]{}
\makeatletter

\usepackage[UTF8]{inputenc}
%\usepackage[T1]{fontenc}
%\usepackage[ngerman]{babel}

\usepackage{comment}

\usepackage{tabularx}



\linespread{1.6} %%%%%%%%%%%%%%%%%%%%% Used for drafting %%%%%%%%%%%%%%%%%%%%%
\usepackage[margin=2.5cm]{geometry} %%%%%%%%%%%%%%%%%%%%% Used for drafting %%%%%%%%%%%%%%%%%%%%%

%=====================================================================================

\begin{document} 
%
\frontmatter          % for the preliminaries 
%
\pagestyle{headings}  % switches on printing of running heads
% \addtocmark{Hamiltonian Mechanics} % additional mark in the TOC
%

\author{David Eigenstuhler BA}
\title{Augmented Reality Support for Business Process Modeling}
\titlerunning{Augmented Reality in BPM}  % abbreviated title (for running head)
%                                     also used for the TOC unless
%                                     \toctitle is used
%\subtitle{Proposal}
\institute{Communications Engineering\\
Department of Business Information Systems\\
Johannes Kepler University of Linz}
\maketitle

%\vspace{5cm} 
\begin{flushright}\noindent
April 2014 
\end{flushright}
\vspace{1cm}

%\begin{tabular}{p{8cm}p{8cm}}
%\begin{tabular}{>{\centering\arraybackslash}p{4cm}p{4cm}p{4cm}}
\begin{tabular}{p{4cm}p{4cm}p{4cm}}
Course Supervisor & & Course Supervisor \tabularnewline 
Udo Kannengiesser & & Eric Brewster \tabularnewline 
\end{tabular} 

%=====================================================================================

%\newpage

\vspace{1cm}
\begin{abstract}
Using virtual worlds or Augmented Reality for Business Process Modeling (BPM) is a fairly new approach 
in the process modeling field. Having domain experts with little modeling skills and process modeling
experts with no knowledge of the process to be modeled requires new solutions. This paper examines three
different methods to give an overview of current approaches, which try to overcome this problem by
tangible physical objects, virtual reality, and Augmented Reality respectively. The analysis is done
on the basis of secondary research only. Physical and tactile modeling makes it easy for domain experts 
to interact with the model. The major drawback of the necessary manual transcription of the results is 
being avoided by the idea to include the modeler into virtual worlds. Using the Second Life engine provides
easier immersion into the model than general BPM methods. This solution lacks the natural communication
component such as nonverbal signals, which is overcome by combining the two aforementioned solutions. 
An augmented tabletop featuring physical objects and direct software integration requires in turn special
hardware. There is no perfect method at this time but current methods already provide significant advantages
over traditional BPM procedures.\\
\vspace{1cm}
 
\textbf{Keywords: }
\textit{Business Process Modeling, Augmented Reality, Virtual Reality,
Tangible-BPM, Subject-oriented BPM}
\end{abstract}

%=====================================================================================
%
%\tableofcontents
%
\mainmatter              % start of the contributions
%
\title{Augmented Reality Support for Business Process Modeling}
%
%=====================================================================================

\section{Introduction}

The aim of this term paper is to give an overview of the potential of collaborative Business Process Modeling (BPM) combined with Augmented 
Reality methods. The paper compares current approaches with physical objects, virtual worlds and augmented two-dimensional (2D) visualizations. 


To comprehend the advantages and disadvantages of the application of collaborative modeling in a virtual environment, 
such as Augmented Reality, one has to understand the process of business process modeling and the resulting product first. 
\citeA{Frederiks2004} interpret the business process modeling as a procedure, consisting of different key phases. 
These key stages are iterative and repeat in cycles between contributing stakeholders. This interaction brings up two problems, 
which have to be solved in order to avoid discrepancies in compatibility, efficiency and quality. 


Physical distribution of the collaborators complicates the interaction and communication between them. \citeA{Benford2001} outline
the evolution of IT supported communication forms to face this problem. Remote collaboration started by electronic mails without 
attachments evolved into multimedia forms including pictures, sound and video and mature into collaborative virtual environments (CVE).
However, \citeA{Benford2001} only describe the distribution solution via CVE from a technical point of view, whereas this paper
focuses on the use of CVE in business process modeling. 


The second problem mentioned above is the concurrency of the individual stakeholder's modeling. Complex business processes demand 
the integration of different stakeholders, mainly domain experts. As a usual practice, a workshop with all necessary collaborators 
will be held. Obvious disadvantages are the aforementioned distribution and a sequential working style. This practice leads to 
inefficiency, as the participants will be idling most of the time. To still keep the effective workshop communication while taking 
advantage of an efficient parallel working style, new IT supported systems have to be used. \cite{Nolte2011}


Enabled by todays fast technology and low cost bandwidth, cloud based approaches have emerged. \citeA{Polancic2013} show the 
definite advantages in simplicity and efficiency of cloud based modeling tools over desktop based tools for performing collaborative 
business process modeling between individual stakeholders. \citeA{Brown2011} argue that two-dimensional (2D) virtual
collaborative environments still lack deeper immersion with the objects and get too confusing with annotations, needed for less 
tech-savvy domain experts. To overcome these restrictions a three-dimensional (3D) environment is needed. Within a spatial virtual 
world, annotation are, first of all, less needed and second of all, better placeable. The greatest advantage is consequently the vivid 
depiction of the business process model.


There are a few approaches to merge business process modeling with the use of CVE. Existing literature focuses primarily on one
solution or tool. This paper takes several studies and compares them regarding advantages and disadvantages for business process 
modeling in a virtual collaborative environment.

\section{Context}

Process modeling requires communication between numerous persons, like business analysts, domain experts or other stakeholders. 
While different process modeling tools and notations are available, most of them are missing the possibility for collaboration. 
Moreover, only business analysts use these business process modeling tools. Augmented Reality and the application of virtual 
environments have the capability to improve the collaboration possibilities and 
to simplify process modeling. 

\section{Evidence and Analytical Methods} 

Different kinds of secondary literature are searched and evaluated in research libraries provided by the University Linz. 
The primary resources are Emerald, ScienceDirect, IEEEXplore, ACM and Springer Link. All search queries are documented in a 
separate list, containing the search term, the source and some optional notes.

\section{Business Process Modeling}

This section gives a short overview of the purpose of BPM.\\
\cite{Davenport1993}\\
\cite{Gou2000}\\
\cite{Loos2008}\\
\cite{Aalst2000}

	\subsection{The Problem: Collaboration Support and Complexity}
	Problems regarding business process modeling in organizations and difficulties concerning the understanding of BPM tools and notations are explained.\\
	\cite{Cognini2013} \\
	\cite{Backer2009}\\
	\cite{Hahn2010}\\
	\cite{Nolte2011}\\
	\cite{Polancic2013}
	
	\subsection{The Solution: Augmented Reality Approaches for BPM}
	The benefits of Augmented Reality support for BPM are described.
	
\section{Current Approaches and Technologies}
This segment deals with the evaluation of current approaches and technologies to support collaborative BPM with added dimensions
to common BPM. It starts with a tangible and IT-unsupported approach through the Tangible BPM tookit. The next technology 
shows BPM in Virtual Worlds. Process models are represented in a three dimensional area, realized through the Second Life 
engine. An analysis of an Augmented Reality tabletop closes the section.

	\subsection{Tangible BPM}
	
	Visualizing business process models helps creating a shared understanding in busines 
	process management. Furthermore, \citeA{Grosskopf2009} argues that only few domain experts
	are able to create and read a process model. This leads to limited feedback and error correction
	for the model from the method experts who have the required business knowledge. Representing the
	models more figurative aids all participants to comprehend the process illustration more easily.

	\citeA{Luebbe2011} describes the objective for easier process-orientated communication amongst
	domain and BPM experts through Tangible BPM (TBPM) as designing an approach to engage domain
	experts in the creation and validation of their process models. This is realized by a TBPM 
	toolset consisting of four basis shapes representing activities, events, gateways, and data
	objects. According to \citeA{Luebbe2011}, Tangible stands for the tactile part of touching and 
	interacting manually with these	physical items. The objects have their direct counterpart in 
	Business Process Model and Notation (BPMN), a widely used BPM language. By writing on the items,
	nearly all of over fifty process modeling shapes in the BPMN 2.0 standard can be representated. 
	Consequently, process models created with the TBPM toolkit comply with the BPMN 2.0 standard 
	while being more intuitive and expressive. The BPM expert changes his role from a modeler to
	an facilitator, where he is in an auxiliary position, in contrast to his former executing
	function.

	\citeA{Edelman2009} did some pilot studies, where they compared business process elicitation
	through structured interviews, which is the accepted method to date, and interviews supported by TBPM. Through their
	experiments they formed four preliminary hypotheses. Domain experts have a higher consensus	%%%%%%%%% H05
	with the produces model and identify more with it. Furthermore, self-correction of			%%%%%%%%% H07
	misunderstanding and mistakes is higher. By the hands-on modeling experience, process experts %%%%%%% H09
	gain a better understanding of the process itself. Their last hypothesis states, that
	interviewers remember more details, because of the additional dimensions of visual perception and
	tactile sensation. 

	\citeA{Luebbe2011} evaluated among others the first three above mentioned hypotheses with a 
	controlled experiment of 17 students. He came to the conclusion that, first of all, TBPM 
	activates people. The inclination to emerge with the task of process elication is higher than
	with general interviews. \citeA{Grosskopf2009} remark, that the analogy to children's blocks 
	significantly lowered the barrier of domain experts to interact with the model. Participants take 
	more time to talk about the process and use longer periods to think about it. This leads therefore 
	to higher consensus with the resulting model.
	\citeA{Luebbe2011} also came to the conclusion, that domain experts using TBPM review the model more
	often and correct more errors, which are inevitable in structured interviews with or without
	TBPM. This results to more validated process models. The hypothesis stated by \citeA{Edelman2009},
	that the hands-on modeling leads to better understanding of the process experts own process
	was also verified by \citeA{Luebbe2011}.

	Being physically co-located makes it easier to collaborate with more than one domain expert. Communication is done face
	to face and important nonverbal signals are transmitted as well. Through the involvement of the subject-matter expert (SME)
	and deeper immersion into the model via tactile objects the respective specialist associates more with the
	created model. 
	As elicitation via the TBPM method is done without the help of IT (information technology), the results have to be 
	manually transferred into a digital business process model, which is a major drawback. This also greatly reduces the
	use of T-BPM for process optimizing and adapting, as the models have to be manually laid out before the adapting 
	can begin. \\	
	
	
	\subsection{Augmented Reality for Collaborative BPM}

	\citeA{Benford2001} emphazise on the importance of computers and networks to be utilized to support collaborative work 
	and social play. Collaborative Virtual Environments (CVEs) are Virtual Worlds which are taken advantage of by multiple
	subjects. The participating actors are embodied by avatars in the CVE and can freely move around, interact among themselves,
	and interplay with objects and entities within the world. Communication can be realized using different media including
	text, audio, video, visual gestures, and artefacts. According to \citeA{Brown2010}, the main point lacking in regards to
	BPM is an approach to combine CVEs with conceptual modeling, which means, integrating standardized BPM grammar into an
	intuitive Virtual World. In their paper, they come up with a prototype framework which joins the best from both worlds;
	conceptual BPM and virtual simulation of processes within their context.
	
	The framework following \citeA{Brown2010} allows to facilitate the process modeling and the communication towards
	the viewer. Crossing the boundaries between traditional BPM and virtual simulation empowers the domain experts to
	take part in the modeling process and review the model more easily, by creating a common base for domain and BPM experts.
	\citeA{Brown2010} introduces a framework based on the open source Second Life viewer, which provides simple usage. 
	The game Second Life itself is frequented with nearly 100,000 users a day, of which a large porpotion has learned 
	by itself to use the program and needed no external support.
	
	\citeA{Brown2010} explicate, that the objects within the CVE, which represent conceptual counterparts of BPMN, are Events, Activities,
	Gateways, and Flows. They are displayed as 3D items with connections nodes to enable linking to other objects. Selected
	textures accompany the shapes to indicate different meanings. Next to flow connections, avatars representing
	human actors are enhancing the virtual immersion. Signposts can hold text, images, audio or video which is used 
	as annotation in proximity to an annotated BPM object.
	
	As communication is a very interactive process, the use of tools following \citeA{Brown2010} can facilitate the 
	presentation and adaption of the model among several participants. Communication over networks is enhanced using
	3D avatars compared to traditional means. Dealing with situation where the partners are physically distributed,
	utilizing avatars brings a new dimension to CVEs with information such as where they stand, which direction they 
	face or which path they are taking.
	
	A case study by \citeA{Brown2011} on using the tool aforementioned by \citeA{Brown2010} comes to the conclusion, that
	utilizing Virtual Worlds for collaborative BPM can contribute to improved communication in distributed situations.
	Interaction among physically dispersed partners is superior compared to commonly used textual and audio-visual methods, 
	as mannerisms and other non-verbal messages are also transported and interpreted, although in a limited way. The learning
	effort to use the framework is fast and manageable, especially with a facilitator who supports the modeling and reviewing
	participants. 
	
	\subsection{Comprehand}
		
	\citeA{Oppl2011} describes in his paper a tabletop interface for business process modeling developed at the University
	of Linz. This approach uses Subject-oriented Business Process Management (S-BPM) to enable people and organizations to 
	model their business processes on the basis of communication between business entities. Traditional BPM methods focus
	rather on the global control flow then on direct interaction among subjects. \citeA{Oppl2011} furthermore argues, that S-BPM let 
	modelers form their processes from a local starting point, in contrast to a global point of view. Domain experts, who have
	knowledge of the processes from their operational experience find it often hard to externalize their know-how. \citeA{Oppl2011}
	therefore presents an approach, which combines S-BPM with an IT-supported tabletop to facilitate elicitation of operational
	knowledge. This approach takes into consideration, that S-BPM distinguishes between internal behaviour of a subject's core
	process and the interaction among these processes.
	
	\citeA{Oppl2011} states, that elicitating process know-how requires, next to individual knowledge, also a commonly agreed idea
	how to cooperate. The integral part of finding this common understanding has also to be supported by the method to enable interactive
	knowledge elicitation. The tabletop approach described in his paper manages this challenge with co-located and even spatially
	distributed participants. Using the subject-oriented form of process illustration is crucial, because of the separation
	of individual work behaviours and interactions between individual subjects. Up to four domain experts can share a table to
	work in a co-located situation. \citeA{Oppl2011} shows also the possibilty to work with geographically distributed participants 
	by distributing changes of physical models from one location through network broadcasts to the other locations.
	
	The tool for knowledge elicitation from domain experts consists according to \citeA{Oppl2011} of a digitally augmented tabletop system
	with physical modeling shapes. The participants in the process modeling can grasp the objects, move them around and form connections by
	simply putting blocks together. Established connections and concepts can be named and annotated using a keyboard and erased by an
	eraser pen. The model is captured by a camera beneath the table and transformed by an algorithm to its
	digital equivalent. This keeps the digital image synronous with the physical representation. Changes in the digital model
	are displayed on the table display through visual and textual hints, so the modelers can adapt the physical representation to
	stay consistent. All communication between the system and the modeler takes place on the table's display to permit simultaneous
	adaption of the concept map.
	
	The solution of \citeA{Oppl2011} leads to multiple use cases, with co-located and distributed modeling of interaction. For modeling 
	the internal behaviour, it allows for individual or collaborative modeling with co-located or distributed modelers on one or more 
	closely related processes. His paper shows the technical feasibilty of the digitally augmented tabletop approach, which allows
	for process knowledge elicitation with domain experts in physical proximity and geographical dispersion.

		
	\subsection{Comparison of Approaches and Technologies}
	The paper closes with a comparison of the mentioned approaches and technologies.
	
\section{Conclusion}

Business process modeling and its underlying notations are very important tools to support
business decisions. Organizations greatly benefit from it by applying BPM for optimizations of processes or work flows.
The substantial problem lays in the usability of model editors and viewers. The actual process of business process modeling is mainly done by business 
process modeling experts (BPM experts). These BPM professionals in turn usually lack profound knowledge 
of the processes to be modeled. To overcome this discrepancy the modelers need the help of domain experts, 
who work in the respective field. As BPM models are hard to understand, build, and adapt, functional experts
need the help of modeling experts to transfer their knowledge into the models. The usual work method is to transmit
the functional expertise in interviews or workshops.


Tangible BPM (T-BPM) provides an approach to overcome this problem
by adding physical objects T-BPM manages to simplify the abstract model building process. 
%Four differently shaped objects included in the T-BPM toolbox stand for their conceptual counterparts in the BPM Notation (BPMN). 
Domain experts are now able to work directly with the model, as the modeling is more intuitive and only a short training 
period is needed. BPM experts act as facilitators to moderate the procedure and support the participants.
Collaboration with more than one domain expert is facilitated and precious nonverbal signals do not get lost through 
digital conversion. Manual transcription of the resulting models form nonetheless a major disadvantage to IT-supported
BPM modeling processes. 


To avoid the disadvantage of manual transcription another approach has been developed. Instead of bringing the model
into a physical form this attempt tries to bring the modeler into a digital form. \citeA{Brown2011} describe a prototype
to realize collaborative BPM within a virtual world. They use a modified version of the open source engine Second Life.
The modelers are presented as avatars within the 3D environment. BPM objects like functions, states, or annotations are
displayed in an intuitive way and can easily be modified by the avatars. \\
As with T-BPM, the process of BPM is simplified and made more intuitive. Therefore, domain experts can also directly use this method.
The aforementioned downside of manual transcription is completely avoided, as the whole process is taking place within
the software, which itself transcribes the data into a BPM conform language. As a result of this situation, collaboration
is not limited to co-located participants. Given that every partner is provided with a laptop, the needed software, and an Internet 
connection, there are no limitations to cooperate even within a spatially distributed group. Communication is realized via text
messages, annotations within the 3D map, and audio and video (A/V) conferences. Compared to T-BPM, the virtual world solution faces the 
problem of losing nonverbal information through digital conversations. Because of the digital immersion into the model, even 
co-located modeling sessions are confronted with this problem. 


A new method to get the best out of both worlds is presented by \citeA{Oppl2011} with an augmented tabletop. This approach combines
the tactile features of T-BPM with the IT based functionality of virtual worlds. A display functions as the table plate and 
illustrates connections, annotations, and auxiliary information. The placed physical objects are captured by a camera from 
beneath and interpreted by an algorithm. Domain experts find themselves at ease to draft and edit business process models, 
while all actions are directly captured in digital form. There is no need to manually translate the results in 
a computer readable format. Having the option to co-work on a single tabletop, communication stays natural and direct.
If the situation requires physically distributed collaboration, this method allows remote communication and synchronizing 
through the Internet or private networks. A major disadvantage of this augmented reality solution is the
need of proprietary hardware.


None of the methods examined in this paper offer a perfect solution. Either the workload around the modeling is increased,
communication and ways of collaboration are more complicated or the requirements are higher. One has to keep in mind, 
that Collaborative BPM with augmented reality (AR) support is still a new field and these approaches are just the first steps
in a new direction. Nonetheless, even now, the reviewed solutions already offer significant advantages in some fields to general BPM.




\begin{comment}
\section{Outline} 

The following outline gives an overview of the term paper:
\begin{itemize}
	\item \textbf{Abstract}\\
	The abstract summarizes content and motivation of the paper.
	\vspace{3mm}
	
	\item \textbf{Introduction}\\
	The core concepts are introduced.
	\vspace{3mm}
	
	\begin{itemize}
		\item \textbf{Business Process Modeling}\\
		This section gives a short overview of the purpose of BPM.\\
		\cite{Davenport1993}\\
		\cite{Gou2000}\\
		\cite{Loos2008}\\
		\cite{Aalst2000}
		\vspace{3mm}
				
		\item \textbf{The Problem: Collaboration Support and Complexity}\\
		Problems regarding business process modeling in organizations and difficulties concerning the understanding of BPM tools and notations are explained.\\
		\cite{Cognini2013} \\
		\cite{Backer2009}\\
		\cite{Hahn2010}\\
		\cite{Nolte2011}\\
		\cite{Polancic2013}
		\vspace{3mm}
		
		\item \textbf{The Solution: Augmented Reality Approaches for BPM}\\
		The benefits of Augmented Reality support for BPM are described.
		\vspace{3mm}
		
	\end{itemize}
	\item \textbf{Current Approaches and Technologies} \\
	This segment deals with the evaluation of current approaches and technologies to support BPM with Augmented Reality.
	\vspace{3mm}
	
		\begin{itemize}
		\item \textbf{Tangible BPM}\\
		Tangible BPM is a solution that uses plastic cards for process modeling.
		\cite{Edelman2009}\\
		\cite{Grosskopf2009}\\
		\cite{Luebbe2011}
		\vspace{1mm}
		
		\begin{itemize}
			\item Introduction
			\item Process Modeling Concept
			\item BPM Standard Support
			\item Tool Support
			\item Pros and Cons
		\end{itemize}
		\vspace{3mm}
		
		\item \textbf{Augmented Reality for Collaborative BPM}\\
		A solution is proposed that combines the benefits of the virtual world Second Life with the advantages of Augmented Reality. \\
		\cite{Brown2010}\\
		\cite{Brown2011}\\
		\cite{Poppe2011}\\
		\cite{Regenbrecht2006}\\
		\cite{Benford2001}
		\vspace{1mm}

		\begin{itemize}
			\item Introduction
			\item Pros and Cons
		\end{itemize}
		\vspace{3mm}

		\item \textbf{Comprehand} \\
		Comprehand is a table-top interface business process modeling tool developed at the University of Linz.
		\vspace{1mm}
		
		\begin{itemize}
			\item Introduction
			\item Pros and Cons
			\item Augmented Reality for Subject-Oriented BPM\\
			\cite{Oppl2011}
			\item Subject-Oriented BPM\\
			\cite{Oppl2011}
			\item Metasonic Touch
		\end{itemize}
		\vspace{3mm}
		
		\item \textbf{Comparison of Approaches and Technologies} \\
		The paper closes with a comparison of the mentioned approaches and technologies.
		\vspace{3mm}
	\end{itemize}
\item \textbf{Conclusion}\\
The results of the paper are presented in a distilled version.
\end{itemize}
\end{comment}

\clearpage




\printindex

\bibliography{refs}

\clearpage

\pagestyle{empty}
\parindent0mm

David Eigenstuhler BA\\
Johannes Kepler University of Linz, Austria\\
\href{mailto:david@eigenstuhler.at}{David@Eigenstuhler.at}\\
\vspace{4mm}

\textbf{Biodata: } David Eigenstuhler is momentarily attending the Master's program in Business Informatics in Linz. He earned his Bachelor of Arts' degree 
in Business Studies. Next to his studies, he works as an IT administrator at Emesa Austria. Before he commenced to study in 2009, 
he worked in product management and sales promotion.

\end{document}
