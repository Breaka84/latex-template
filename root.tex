% This is LLNCS.DEM the demonstration file of
% the LaTeX macro package from Springer-Verlag
% for Lecture Notes in Computer Science,
% version 2.4 for LaTeX2e as of 16. April 2010
%
\documentclass[12pt]{llncs} %%%%%%%%%%%%%%%%%%%%% Used for drafting %%%%%%%%%%%%%%%%%%%%%
%\documentclass{llncs}
%

\usepackage[colorlinks, linkcolor = black, citecolor = black, filecolor = black, urlcolor = blue]{hyperref} 

% APA Cite Package and initialization
%\RequirePackage{apacite}
\usepackage[apaciteclassic]{apacite}
\bibliographystyle{apacite}

%
\usepackage{makeidx}  % allows for indexgeneration
%
\usepackage{color}
% 
\setcounter{tocdepth}{2}
% 
% make a proper TOC despite llncs
\makeatletter
\renewcommand*\l@author[2]{}
\renewcommand*\l@title[2]{}
\makeatletter

\usepackage[UTF8]{inputenc}
%\usepackage[T1]{fontenc}
%\usepackage[ngerman]{babel}

\usepackage{comment}

\usepackage{tabularx}



\linespread{1.6} %%%%%%%%%%%%%%%%%%%%% Used for drafting %%%%%%%%%%%%%%%%%%%%%
\usepackage[margin=2.5cm]{geometry} %%%%%%%%%%%%%%%%%%%%% Used for drafting %%%%%%%%%%%%%%%%%%%%%

%=====================================================================================

\begin{document} 
%
\frontmatter          % for the preliminaries 
%
\pagestyle{headings}  % switches on printing of running heads
% \addtocmark{Hamiltonian Mechanics} % additional mark in the TOC
%

\author{David Eigenstuhler BA}
\title{Please enter a title}
\titlerunning{Title?}  % abbreviated title (for running head)
%                                     also used for the TOC unless
%                                     \toctitle is used
%\subtitle{Proposal}
\institute{Institute?\\
Department of ???\\
Johannes Kepler University of Linz}
\maketitle

%\vspace{5cm} 
\begin{flushright}\noindent
What date is it?
\end{flushright}
\vspace{1cm}

%\begin{tabular}{p{8cm}p{8cm}}
%\begin{tabular}{>{\centering\arraybackslash}p{4cm}p{4cm}p{4cm}}
\begin{tabular}{p{4cm}p{4cm}p{4cm}}
Course Supervisor & & Course Supervisor \tabularnewline 
??? & & ??? \tabularnewline 
\end{tabular} 

%=====================================================================================

%\newpage

\vspace{1cm}
\begin{abstract}
There goes the abstract\\
\vspace{1cm}
 
\textbf{Keywords: }
\textit{Keyword1, keyword2, keyword3}
\end{abstract}

%=====================================================================================
%
%\tableofcontents
%
\mainmatter              % start of the contributions
%
\title{Title?}
%
%=====================================================================================

\section{Introduction}

Please have a look at \cite{Kopka2003}!

Lorem ipsum dolor sit amet, consetetur sadipscing elitr, sed diam nonumy eirmod
tempor invidunt ut labore et dolore magna aliquyam erat, sed diam voluptua. At

\subsection{sub1}
vero eos et accusam et justo duo dolores et ea rebum. Stet clita kasd gubergren,
no sea takimata sanctus est Lorem ipsum dolor sit amet. Lorem ipsum dolor sit
amet, consetetur sadipscing elitr, sed diam nonumy eirmod tempor invidunt ut

\subsection{sub2}
labore et dolore magna aliquyam erat, sed diam voluptua. At vero eos et accusam
et justo duo dolores et ea rebum. Stet clita kasd gubergren, no sea takimata
sanctus est Lorem ipsum dolor sit amet.

\section{Context}

Lorem ipsum dolor sit amet, consetetur sadipscing elitr, sed diam nonumy eirmod
tempor invidunt ut labore et dolore magna aliquyam erat, sed diam voluptua. At

\subsection{sub1}
vero eos et accusam et justo duo dolores et ea rebum. Stet clita kasd gubergren,

\subsection{sub2}
no sea takimata sanctus est Lorem ipsum dolor sit amet. Lorem ipsum dolor sit
amet, consetetur sadipscing elitr, sed diam nonumy eirmod tempor invidunt ut
labore et dolore magna aliquyam erat, sed diam voluptua. At vero eos et accusam

\subsection{sub3}
et justo duo dolores et ea rebum. Stet clita kasd gubergren, no
sea takimata sanctus est Lorem ipsum dolor sit amet.

\section{Evidence and Analytical Methods} 

Lorem ipsum dolor sit amet, consetetur sadipscing elitr, sed diam nonumy eirmod
tempor invidunt ut labore et dolore magna aliquyam erat, sed diam voluptua. At
vero eos et accusam et justo duo dolores et ea rebum. Stet clita kasd gubergren,
no sea takimata sanctus est Lorem ipsum dolor sit amet. Lorem ipsum dolor sit
\subsection{sub1}
amet, consetetur sadipscing elitr, sed diam nonumy eirmod
tempor invidunt ut labore et dolore magna aliquyam erat, sed diam voluptua. At vero eos et accusam
et justo duo dolores et ea rebum. Stet clita kasd gubergren, no sea takimata
sanctus est Lorem ipsum dolor sit amet.

\section{Main}

Lorem ipsum dolor sit amet, consetetur sadipscing elitr, sed diam nonumy eirmod
tempor invidunt ut labore et dolore magna aliquyam erat, sed diam voluptua. At
vero eos et accusam et justo duo dolores et ea rebum. Stet clita kasd gubergren,
no sea takimata sanctus est Lorem ipsum dolor sit amet. Lorem ipsum dolor sit
amet, consetetur sadipscing elitr, sed diam nonumy eirmod tempor invidunt ut
labore et dolore magna aliquyam erat, sed diam voluptua. At vero eos et accusam
et justo duo dolores et ea rebum. Stet clita kasd gubergren, no sea takimata
sanctus est Lorem ipsum dolor sit amet.

\section{Conclusion}

Lorem ipsum dolor sit amet, consetetur sadipscing elitr, sed diam nonumy eirmod
tempor invidunt ut labore et dolore magna aliquyam erat, sed diam voluptua. At
vero eos et accusam et justo duo dolores et ea rebum. Stet clita kasd gubergren,
no sea takimata sanctus est Lorem ipsum dolor sit amet. Lorem ipsum dolor sit
amet, consetetur sadipscing elitr, sed diam nonumy eirmod tempor invidunt ut
labore et dolore magna aliquyam erat, sed diam voluptua. At vero eos et accusam
et justo duo dolores et ea rebum. Stet clita kasd gubergren, no sea takimata
sanctus est Lorem ipsum dolor sit amet.



\clearpage




\printindex

\bibliography{bib/library}

\clearpage

%\pagestyle{empty}
\parindent0mm

David Eigenstuhler BA\\
Johannes Kepler University of Linz, Austria\\
\href{mailto:david@eigenstuhler.at}{David@Eigenstuhler.at}\\
\vspace{4mm}

\textbf{Biodata: } David Eigenstuhler is momentarily attending the Master's program in Business Informatics in Linz. He earned his Bachelor of Arts' degree 
in Business Studies. Next to his studies, he works as an IT administrator at Emesa Austria. Before he commenced to study in 2009, 
he worked in product management and sales promotion.

\end{document}
